\documentclass{beamer}
\usepackage{amsmath, amssymb}
\usepackage{bm}

\begin{document}

%%%%%%%%%%%%%%%%%%%%%%%%%%%%%%%%%%%%%%%%%%%%%%%%%%%%%%%%%%%%%%%%%%%%%
% Slide 1: Background and Problem Definition
%%%%%%%%%%%%%%%%%%%%%%%%%%%%%%%%%%%%%%%%%%%%%%%%%%%%%%%%%%%%%%%%%%%%%
\begin{frame}
\frametitle{Background and Problem Definition}

\textbf{1. Setting: Discrete Time Integration}
\begin{itemize}
    \item We have $m$ particles in $\mathbb{R}^3$ (so $3m$ coordinates in total). Time is discretized into steps of size $h$, so the system's state at step $n$ is $\mathbf{q}_n\in\mathbb{R}^{3m}$.
    \item The forces are given by
    \[
    \mathbf{f}(\mathbf{q})=-\nabla E(\mathbf{q}),
    \]
    where $E$ is a (generally non-linear, possibly non-convex) potential energy and $\mathbf{M}$ is a (diagonal) mass matrix.
\end{itemize}

\bigskip

\textbf{2. Implicit Euler Update Equations}
\begin{itemize}
    \item \textbf{Position update:}
    \[
    \mathbf{q}_{n+1}=\mathbf{q}_n+h\,\mathbf{v}_{n+1}
    \]
    \item \textbf{Velocity update:}
    \[
\mathbf{v}_{n+1}=\mathbf{v}_n+h\,\mathbf{M}^{-1}\mathbf{f}\big(\mathbf{q}_{n+1}\big)
    \]
\end{itemize}

the velocity at the next time step depends on the forces evaluated at the new position, hence the term ``implicit Euler.''

\end{frame}

\begin{frame}
\frametitle{Background and Problem Definition}

\textbf{3. Eliminating Velocities}
\begin{itemize}
    \item Using
    \[
    h\,\mathbf{v}_n=\mathbf{q}_n-\mathbf{q}_{n-1} \quad \text{and} \quad h\,\mathbf{v}_{n+1}=\mathbf{q}_{n+1}-\mathbf{q}_n,
    \]
    we substitute into the velocity update to obtain:
    \[
    \mathbf{q}_{n+1}-2\mathbf{q}_n+\mathbf{q}_{n-1}=h^2\,\mathbf{M}^{-1}\mathbf{f}\big(\mathbf{q}_{n+1}\big) \quad (5).
    \]
    \item This is a discrete version of Newton's $\mathbf{F}=\mathbf{M}\mathbf{a}$ (with acceleration approximated by
    $  \frac{\mathbf{q}_{n+1}-2\mathbf{q}_n+\mathbf{q}_{n-1}}{h^2}.$)
    \item Equation (5) is nonlinear and is typically solved with a Newton (or quasi-Newton) method. (Recall the classical approach by Baraff and Witkin (1998) that linearizes the force as 
    \[
    \mathbf{f}(\mathbf{q}_{n+1})\approx \mathbf{f}(\mathbf{q}_n)+\nabla\mathbf{f}\big|_{\mathbf{q}_n}(\mathbf{q}_{n+1}-\mathbf{q}_n).
    \]
    
\end{itemize}

\end{frame}


\begin{frame}
\frametitle{Background and Problem Definition}
\textbf{4. Reformulating as an Optimization Problem}
\begin{itemize}
    \item Define:
    \[
    \mathbf{x}:=\mathbf{q}_{n+1}, \quad \mathbf{y}:=2\mathbf{q}_n-\mathbf{q}_{n-1}.
    \]
    Then equation (5) becomes:
    \[
    \mathbf{M}(\mathbf{x}-\mathbf{y})=h^2\,\mathbf{f}(\mathbf{x}) \quad (7).
    \]
    \item Since $\mathbf{f}(\mathbf{x})=-\nabla E(\mathbf{x})$, we have:
    \[
    \mathbf{M}(\mathbf{x}-\mathbf{y})+h^2\,\nabla E(\mathbf{x})=0.
    \]
    \item Notice that this is the gradient of the function
    \[
    g(\mathbf{x})=\frac{1}{2}(\mathbf{x}-\mathbf{y})^T\mathbf{M}(\mathbf{x}-\mathbf{y})+h^2\,E(\mathbf{x}) \quad (8).
    \]
    with respect to $\mathbf{x}$. 


    \end{itemize}
\end{frame}


\begin{frame}
\frametitle{Background and Problem Definition}
So, we have:
$$
\nabla g(\mathbf{x})=\mathbf{M}(\mathbf{x}-\mathbf{y})+h^2 \nabla E(\mathbf{x})
$$


Hence, setting $\nabla g(\mathbf{x})=0$ is equivalent to solving $\mathbf{M}(\mathbf{x}-\mathbf{y})=h^2 \mathbf{f}(\mathbf{x})$.

Therefore:
Implicit Euler is the same as solving the optimization problem

$$
\min _{\mathbf{x}} \underbrace{\frac{1}{2}(\mathbf{x}-\mathbf{y})^T \mathbf{M}(\mathbf{x}-\mathbf{y})+h^2 E(\mathbf{x})}_{=: g(\mathbf{x})}
$$

But we still need Newton or a similar method to minimize $g(\mathbf{x})$.

    \textbf{Connection to Position-Based Dynamics (PBD):}\\
    If we define a potential energy $E_{PBD}$ (whose terms are the squared constraint violations), then PBD effectively tries to reduce
    \[
    g(\mathbf{x})=\frac{1}{2}(\mathbf{x}-\mathbf{y})^T\mathbf{M}(\mathbf{x}-\mathbf{y})+h^2\,E_{PBD}(\mathbf{x})
    \]
    in a Gauss-Seidel-like solver.


\end{frame}

%%%%%%%%%%%%%%%%%%%%%%%%%%%%%%%%%%%%%%%%%%%%%%%%%%%%%%%%%%%%%%%%%%%%%
% Slide 3: Main Part -- Hooke's Law and the Reformulation
%%%%%%%%%%%%%%%%%%%%%%%%%%%%%%%%%%%%%%%%%%%%%%%%%%%%%%%%%%%%%%%%%%%%%
\begin{frame}
\frametitle{Main Part: Hooke's Law and the Reformulation}

\textbf{Hooke's Law (Equation 9)}
\begin{itemize}
    \item A single spring connecting points $\mathbf{p}_1$ and $\mathbf{p}_2\in\mathbb{R}^3$ with rest length $r$ has the potential:
    \[
    \frac{1}{2}k\Bigl(\|\mathbf{p}_1-\mathbf{p}_2\|-r\Bigr)^2,
    \]
    where $k$ is the spring stiffness.
    \item \textbf{Reformulation:} Introduce an auxiliary vector $\mathbf{d}\in\mathbb{R}^3$ with $\|\mathbf{d}\|=r$. Then,
    \[
    \Bigl(\|\mathbf{p}_1-\mathbf{p}_2\|-r\Bigr)^2=\min_{\|\mathbf{d}\|=r}\Bigl\|\bigl(\mathbf{p}_1-\mathbf{p}_2\bigr)-\mathbf{d}\Bigr\|^2.
    \]

\end{itemize}

\end{frame}

\begin{frame}{proof of reformulation}
\small
    Lemma. For each $\mathbf{p}_1, \mathbf{p}_2 \in \mathbb{R}^3$ and $r \geq 0$ :

$$
\left(\left\|\mathbf{p}_1-\mathbf{p}_2\right\|-r\right)^2=\min _{\|\mathbf{d}\|=r}\left\|\left(\mathbf{p}_1-\mathbf{p}_2\right)-\mathbf{d}\right\|^2
$$


Proof. For brevity we define $\mathbf{p}_{12}:=\mathbf{p}_1-\mathbf{p}_2$. Given the constraint $\|\mathbf{d}\|=r$, we rewrite the left side of the equation:

$$
\left(\left\|\mathbf{p}_{12}\right\|-r\right)^2=\left(\left\|\mathbf{p}_{12}\right\|-\|\mathbf{d}\|\right)^2
$$


By applying the reverse triangle inequality, we have:

$$
\left(\left\|\mathbf{p}_{12}\right\|-\|\mathbf{d}\|\right)^2 \leq\left\|\mathbf{p}_{12}-\mathbf{d}\right\|^2
$$


Next, if we substitute $\mathbf{d}=r\left(\mathbf{p}_{12} /\left\|\mathbf{p}_{12}\right\|\right)$ to the right side, we obtain:

$$
\left\|\mathbf{p}_{12}-r \frac{\mathbf{p}_{12}}{\left\|\mathbf{p}_{12}\right\|}\right\|^2=\left\|\frac{\mathbf{p}_{12}}{\left\|\mathbf{p}_{12}\right\|}\left(\left\|\mathbf{p}_{12}\right\|-r\right)\right\|^2=\left(\left\|\mathbf{p}_{12}\right\|-r\right)^2
$$


Therefore, when $\mathbf{d}=r\left(\mathbf{p}_{12} /\left\|\mathbf{p}_{12}\right\|\right)$, the right hand side of the equation produces its minimum value that equals to the leff.
\end{frame}

%%%%%%%%%%%%%%%%%%%%%%%%%%%%%%%%%%%%%%%%%%%%%%%%%%%%%%%%%%%%%%%%%%%%%
% Slide 4: Main Part -- Summing Over All Springs \& Matrix Form
%%%%%%%%%%%%%%%%%%%%%%%%%%%%%%%%%%%%%%%%%%%%%%%%%%%%%%%%%%%%%%%%%%%%%
\begin{frame}
\frametitle{Main Part I: Summing Over All Springs}

\begin{itemize}
    \item Sum the spring potentials over all springs (indexed by $\mathbf{p_i}, i=1,\ldots,s$). For each spring $i$ connecting particles with indices $(i_1,i_2)$, we have endpoints $\mathbf{p}_{i_1}$ and $\mathbf{p}_{i_2}$.
    \item Stack all positions into a single vector:
    \[
    \mathbf{x}=(\mathbf{p}_1,\mathbf{p}_2,\ldots,\mathbf{p}_m)\in\mathbb{R}^{3m}.
    \]
    \item With the spring direction vectors $\mathbf{d}_i\in\mathbb{R}^3$ with $\|\mathbf{d}_i\|=r_i$, and the single spring has the potential $\frac{1}{2}k_i\Bigl(\|\mathbf{p}_{i1}-\mathbf{p}_{i2}\|-r_i\Bigr)^2,$ the overall spring energy becomes
    \[
    \frac{1}{2}\sum_{i=1}^s k_i\Bigl\|\mathbf{p}_{i_1}-\mathbf{p}_{i_2}-\mathbf{d}_i\Bigr\|^2.
    \]
    \item This can be rewritten in the form
    \[
    \frac{1}{2}\mathbf{x}^T\mathbf{L}\mathbf{x}-\mathbf{x}^T\mathbf{J}\mathbf{d} + (\text{constant}) (11),
    \]
\end{itemize}
\end{frame}

\begin{frame}{Main Part II: Matrix Form}
\small
\textbf{Incidence Vectors $\mathbf{A}_i$ to express $\mathbf{p}_{i1}-\mathbf{p}_{i2}$}

$\mathbf{A}_i \in \mathbb{R}^m$ (the shape depends on the number of particles)is the incidence vector between particles $i_1$ and $i_2$.
Specifically,
$\left(\mathbf{A}_i\right)_{i_1}=+1$,
$\left(\mathbf{A}_i\right)_{i_2}=-1$,
and 0 elsewhere.

Consider a one-dimensional version of the particle positions $\mathbf{x}_{\text {(scalar) }}$. In the full 3D setting, we have $\mathbf{x}=\left(\mathbf{p}_1, \mathbf{p}_2, \ldots, \mathbf{p}_m\right) \in \mathbb{R}^{3 m}$
with each $\mathbf{p}_i \in \mathbb{R}^3$. In a one-dimensional system, each particle's position is just a single number. So the analogous one-dimensional position vector would be $\mathbf{x}_{\text {(scalar) }}=\left(x_1, x_2, \ldots, x_m\right) \in \mathbb{R}^m$, where each $x_i$ is the coordinate of particle $i$ along the axis.
so when we say $\left(\mathbf{A}_i\right)^{\top} \mathbf{x}_{(\text {scalar })}=x_{i_1}-x_{i_2}$

Hence,
$$
\left(\mathbf{A}_i\right)^{\top} \mathbf{x}_{\text {(scalar) }}=x_{i_1}-x_{i_2} \quad \text { in 1D. }
$$


In 3D, we do the same but replicate this via a Kronecker product with the $3 \times 3$ identity, so that we have $\left(\mathbf{A}_i \otimes \mathbf{I}_3\right): \mathbb{R}^{3 m} \rightarrow \mathbb{R}^3$, and$
\left(\mathbf{A}_i \otimes \mathbf{I}_3\right) \mathbf{x}=\mathbf{p}_{i_1}-\mathbf{p}_{i_2}$


Therefore,
$
\mathbf{p}_{i_1}-\mathbf{p}_{i_2}=\left(\mathbf{A}_i \otimes \mathbf{I}_3\right) \mathbf{x}
$

\end{frame}

\begin{frame}{Main Part II: Matirx Form}
\textbf{Rewriting $\left\|\mathbf{p}_{i_1}-\mathbf{p}_{i_2}-\mathbf{d}_i\right\|^2$}

Thus, for spring $i$ :

$$
k_i\left\|\mathbf{p}_{i_1}-\mathbf{p}_{i_2}-\mathbf{d}_i\right\|^2=k_i\left\|\left(\mathbf{A}_i \otimes \mathbf{I}_3\right) \mathbf{x}-\mathbf{d}_i\right\|^2
$$


When we sum over $i=1, \ldots, s$, we get

$$
\sum_{i=1}^s k_i\left\|\mathbf{p}_{i_1}-\mathbf{p}_{i_2}-\mathbf{d}_i\right\|^2=\sum_{i=1}^s k_i\left\|\left(\mathbf{A}_i \otimes \mathbf{I}_3\right) \mathbf{x}-\mathbf{d}_i\right\|^2
$$


We can expand each squared norm:

$$
k_i\left[\left(\left(\mathbf{A}_i \otimes \mathbf{I}_3\right) \mathbf{x}\right)^{\top}\left(\mathbf{A}_i \otimes \mathbf{I}_3\right) \mathbf{x}-2\left(\left(\mathbf{A}_i \otimes \mathbf{I}_3\right) \mathbf{x}\right)^{\top} \mathbf{d}_i+\left\|\mathbf{d}_i\right\|^2\right]
$$


Summing over $i$, we can rearrange the terms into:
1. A quadratic form in $\mathbf{x}$.
2. A cross term in $\mathbf{x}$ and $\mathbf{d}_i$.
3. A term purely in $\mathbf{d}_i$.
\end{frame}

\begin{frame}{Main Part II: Matirx Form}
\small
\textbf{Collecting Terms into $\mathbf{L}$ and $\mathbf{J}$}
Quadratic Form ( $\mathbf{x}^{\top} \mathbf{L x}$ )

For the part in $\mathbf{x}$ alone,

$$
\sum_{i=1}^s k_i\left(\mathbf{A}_i \otimes \mathbf{I}_3\right)^{\top}\left(\mathbf{A}_i \otimes \mathbf{I}_3\right)=\left(\sum_{i=1}^s k_i \mathbf{A}_i \mathbf{A}_i^{\top}\right) \otimes \mathbf{I}_3
$$


Define the stiffness-weighted Laplacian in the scalar domain ( $m \times m$ ):

$$
\underbrace{\sum_{i=1}^s k_i \mathbf{A}_i \mathbf{A}_i^{\top}}_{\text {scalar Laplacian }} \in \mathbb{R}^{m \times m} .
$$


Then replicate it in 3D with the Kronecker product to form

$$
\mathbf{L}=\left(\sum_{i=1}^s k_i \mathbf{A}_i \mathbf{A}_i^{\top}\right) \otimes \mathbf{I}_3 \in \mathbb{R}^{3 m \times 3 m}
$$




\end{frame}

\begin{frame}{Main Part II: Matirx Form}
\small
\textbf{Collecting Terms into $\mathbf{L}$ and $\mathbf{J}$(Continue):
The Coupling Matrix $\mathbf{J}\left(-\mathbf{x}^{\top} \mathbf{J} \mathbf{d}\right)$}


Next, for the cross term:
$\sum_{i=1}^s\left(-2 k_i\left(\mathbf{A}_i \otimes \mathbf{I}_3\right) \mathbf{x}^{\top} \mathbf{d}_i\right)$

We want to write this in the form $-2 \mathbf{x}^{\top} \mathbf{J} \mathbf{d}$.
But note that $\mathbf{d}=\left(\mathbf{d}_1, \ldots, \mathbf{d}_s\right)$ is a single vector in $\mathbb{R}^{3 s}$. 

To line up indices, we use indicator vectors $\mathbf{S}_i \in \mathbb{R}^s$, where $\mathbf{S}_i$ is 1 in the $i$-th entry and 0 elsewhere. Then:$
\left(\mathbf{S}_i \otimes \mathbf{I}_3\right) \mathbf{d}=\mathbf{d}_i
$


So the cross term could be written as 
$
\sum_{i=1}^s-2 k_i\left(\mathbf{A}_i \otimes \mathbf{I}_3\right) \mathbf{x}^{\top}\left(\mathbf{S}_i \otimes \mathbf{I}_3\right) \mathbf{d}
$


We can factor out $\mathbf{x}^{\top}$ and $\mathbf{d}$ if we sum:
$
\sum_{i=1}^s-2 k_i\left(\mathbf{A}_i \otimes \mathbf{I}_3\right)\left(\mathbf{S}_i \otimes \mathbf{I}_3\right)^{\top}=-2\left(\sum_{i=1}^s k_i \mathbf{A}_i \mathbf{S}_i^{\top}\right) \otimes \mathbf{I}_3
$


Define the matrix $\mathbb{J}$
$$
\mathbf{J}=\left(\sum_{i=1}^s k_i \mathbf{A}_i \mathbf{S}_i^{\top}\right) \otimes \mathbf{I}_3 \in \mathbb{R}^{3 m \times 3 s}
$$


Then the cross term becomes$-2 \mathbf{x}^{\top} \mathbf{J}\mathbf{d}$ 
\end{frame}

\begin{frame}{Main Part II: Matirx Form}
\small
    \textbf{Wrap it up((Equation 11))}
    Finally, each spring also yields a $\left\|\mathbf{d}_i\right\|^2$ piece, i.e.,

$$
\sum_{i=1}^s k_i\left\|\mathbf{d}_i\right\|^2=\sum_{i=1}^s k_i\left(\mathbf{d}_i\right)^{\top} \mathbf{d}_i=\sum_{i=1}^s k_i\left\|\mathbf{d}_i\right\|^2
$$
which is a constant. 


Combine the three parts:
1. Quadratic in $\mathbf{x}: \frac{1}{2} \mathbf{x}^{\top} \mathbf{L} \mathbf{x}$.
2. Cross term in $\mathbf{x}$ and $\mathbf{d}:-\mathbf{x}^{\top} \mathbf{J} \mathbf{d}$.
3. Constant in $\mathbf{d}: \frac{1}{2} \sum_{i=1}^s k_i\left\|\mathbf{d}_i\right\|^2$. which is usually omit

Hence the total spring energy is
$$
\frac{1}{2} \sum_{i=1}^s k_i\left\|\mathbf{p}_{i_1}-\mathbf{p}_{i_2}-\mathbf{d}_i\right\|^2=\underbrace{\frac{1}{2} \mathbf{x}^{\top} \mathbf{L} \mathbf{x}}_{\text {"Laplacian" term }}-\underbrace{\mathbf{x}^{\top} \mathbf{J} \mathbf{d}}_{\text {"coupling" term }}+\underbrace{\frac{1}{2} \sum_{i=1}^s k_i\left\|\mathbf{d}_i\right\|^2}_{\text {constant}}
$$


\end{frame}

\begin{frame}
\small
\frametitle{Code Helper for the Matrices}

\textbf{Example: Three Particles with Two Springs}
\begin{itemize}
    \item \textbf{Coordinates:} $\mathbf{x}=[\mathbf{p}_1,\mathbf{p}_2,\mathbf{p}_3]^T,\quad \mathbf{p}_i\in\mathbb{R}^3,$,
    so $\mathbf{x}\in\mathbb{R}^9$.
    \item \textbf{Spring Directions:} 
    \[
    \mathbf{d}=[\mathbf{d}_1,\mathbf{d}_2]^T,\quad \mathbf{d}_i\in\mathbb{R}^3,
    \]
    hence $\mathbf{d}\in\mathbb{R}^6$.
    \item \textbf{Incidence and Indicator Vectors:}

        Spring 1 (connects particles 1 and 2):$A_1=\begin{pmatrix} 1 \\ -1 \\ 0 \end{pmatrix}, \quad S_1=\begin{pmatrix} 1 \\ 0 \end{pmatrix}.$   
        \ 
        Spring 2 (connects particles 2 and 3):
        $A_2=\begin{pmatrix} 0 \\ 1 \\ -1 \end{pmatrix}, \quad S_2=\begin{pmatrix} 0 \\ 1 \end{pmatrix}.$
        
\end{itemize}
\end{frame}

\begin{frame}
    \begin{itemize}
    \item \textbf{Constructing $\mathbf{L}$:}
    \begin{enumerate}
        \item Compute $A_iA_i^T$:
        \[
        A_1A_1^T=\begin{pmatrix} 1 & -1 & 0 \\ -1 & 1 & 0 \\ 0 & 0 & 0 \end{pmatrix},\quad
        A_2A_2^T=\begin{pmatrix} 0 & 0 & 0 \\ 0 & 1 & -1 \\ 0 & -1 & 1 \end{pmatrix}.
        \]
        \item Sum with stiffness scaling:
        \[
        k_1A_1A_1^T+k_2A_2A_2^T=
        \begin{pmatrix}
        k_1 & -k_1 & 0 \\
        -k_1 & k_1+k_2 & -k_2 \\
        0 & -k_2 & k_2
        \end{pmatrix}.
        \]
        \item Form the Kronecker product with $I_3$:
        \[
        \mathbf{L}=\Bigl(k_1A_1A_1^T+k_2A_2A_2^T\Bigr)\otimes I_3\in\mathbb{R}^{9\times 9}.
        \]
    \end{enumerate}

    \end{itemize}
\end{frame}

\begin{frame}
\begin{itemize}
    \item \textbf{Constructing $\mathbf{J}$:}
    \begin{enumerate}
        \item Compute $A_iS_i^T$:
        \[
        A_1S_1^T=\begin{pmatrix} 1 & 0 \\ -1 & 0 \\ 0 & 0 \end{pmatrix},\quad
        A_2S_2^T=\begin{pmatrix} 0 & 0 \\ 0 & 1 \\ 0 & -1 \end{pmatrix}.
        \]
        \item Sum with stiffness scaling:
        \[
        k_1A_1S_1^T+k_2A_2S_2^T=
        \begin{pmatrix}
        k_1 & 0 \\
        -k_1 & k_2 \\
        0 & -k_2
        \end{pmatrix}.
        \]
        \item Form the Kronecker product with $I_3$:
        \[
        \mathbf{J}=\Bigl(k_1A_1S_1^T+k_2A_2S_2^T\Bigr)\otimes I_3\in\mathbb{R}^{9\times 6}.
        \]
    \end{enumerate}
\end{itemize}

\textbf{Physical Interpretation:}
\begin{itemize}
    \item $\mathbf{L}$ represents stiffness interactions between particles (diagonal blocks are self-stiffness; off-diagonals represent coupling).
    \item $\mathbf{J}$ couples the spring directions to the particle displacements.
\end{itemize}

\end{frame}

%%%%%%%%%%%%%%%%%%%%%%%%%%%%%%%%%%%%%%%%%%%%%%%%%%%%%%%%%%%%%%%%%%%%%
% Slide 6: Main Part -- Adding External Forces and Defining $E(\mathbf{x})$
%%%%%%%%%%%%%%%%%%%%%%%%%%%%%%%%%%%%%%%%%%%%%%%%%%%%%%%%%%%%%%%%%%%%%
\begin{frame}
\frametitle{Main Part: Adding External Forces and Defining $E(\mathbf{x})$}
\small
\begin{itemize}
    \item With external forces $\mathbf{f}_{\text{ext}}\in\mathbb{R}^{3m}$, a force corresponds to a potential term .
    \item The total potential energy is given by
    \[
    E(\mathbf{x})=\min_{\mathbf{d}\in U}\; \frac{1}{2}\mathbf{x}^T\mathbf{L}\mathbf{x}-\mathbf{x}^T\mathbf{J}\mathbf{d}+\mathbf{x}^T\mathbf{f}_{\text{ext}},
    \]
    where
    \[
    U=\Bigl\{ (\mathbf{d}_1,\ldots,\mathbf{d}_s)\in\mathbb{R}^{3s}:\|\mathbf{d}_i\|=r_i \text{ for each } i\Bigr\}.
    \]
    
\end{itemize}



Recall that implicit Euler leads to minimizing
\[
g(\mathbf{x})=\frac{1}{2}(\mathbf{x}-\mathbf{y})^T\mathbf{M}(\mathbf{x}-\mathbf{y})+h^2\,E(\mathbf{x}) \text{ (equation 8)},
\]
with $\mathbf{y}$ containing known past location.


Substituting the expression for $E(\mathbf{x})$, we have:
\[
\min_{\mathbf{x},\,\mathbf{d}\in U} \; \frac{1}{2}(\mathbf{x}-\mathbf{y})^T\mathbf{M}(\mathbf{x}-\mathbf{y})+h^2\Bigl(\frac{1}{2}\mathbf{x}^T\mathbf{L}\mathbf{x}-\mathbf{x}^T\mathbf{J}\mathbf{d}+\mathbf{x}^T\mathbf{f}_{\text{ext}}\Bigr).
\]

\end{frame}

\begin{frame}
\frametitle{Main Part: Final Objective }
\small
This can be rearranged as:
\[
\min_{\mathbf{x},\,\mathbf{d}\in U} \; \frac{1}{2}\mathbf{x}^T\Bigl(\mathbf{M}+h^2\mathbf{L}\Bigr)\mathbf{x}-h^2\,\mathbf{x}^T\mathbf{J}\mathbf{d}+\mathbf{x}^T\mathbf{b} \text{ (equation 14)},
\]
where $\mathbf{b}$ collects the contributions from the inertia term and $\mathbf{f}_{\text{ext}}$.
Assuming that gravity is the only external force, each particle $i$ has a force $\mathbf{f}_{\mathrm{grav}, i}=m_i \mathbf{g}$


Expanding the inertia term gives a linear part in $\mathbf{x}$ of
$-\mathbf{x}^{\top} \mathbf{M} \mathbf{y}$,
where $\mathbf{y}=2 \mathbf{q}_n-\mathbf{q}_{n-1}$ ( with $\mathbf{x}=\mathbf{q}_{n+1}$ ). So we can have the formula for $\mathbf{b}$ is $\mathbf{b}=-\mathbf{M}\left(2 \mathbf{q}_n-\mathbf{q}_{n-1}\right)+h^2\left(\begin{array}{c}
m_1 \mathbf{g} \\
m_2 \mathbf{g} \\
\vdots \\
m_m \mathbf{g}
\end{array}\right)
$


The whole target function can be solved via the linear system:
\[
\bigl(\mathbf{M}+h^2\mathbf{L}\bigr)\mathbf{x}=h^2\,\mathbf{J}\mathbf{d}-\mathbf{b}.
\]
\end{frame}

\end{document}
