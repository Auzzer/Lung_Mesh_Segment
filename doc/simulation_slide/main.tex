\documentclass[aspectratio=169]{beamer}
\usepackage[utf8]{inputenc}
\usepackage[T1]{fontenc}
\usepackage{lmodern}
\usepackage{amsmath,amssymb,bm}
\usepackage{graphicx}
\usepackage{hyperref}
\usepackage{booktabs}
\usepackage{mathtools}
\usepackage{microtype}
\usepackage[numbers]{natbib}

\graphicspath{{images/}}
\setbeamertemplate{navigation symbols}{}

\begin{document}

% Slide 1
\begin{frame}[t]
  \frametitle{P1 -- A Simple Spring-Mass Analogy (2D cloth)}
  \begin{columns}[T,onlytextwidth]
    \column{0.60\textwidth}
      {\footnotesize
      \textbf{A Simple Spring--Mass Model (2D cloth as intuition)}\\[0.4em]
      \textbf{Mass points \& springs.} Cloth meshes (2D) are often modeled with masses at vertices and linear springs on edges.
       \textbf{Stretch} springs penalize edge length changes; 

       \textbf{shear} springs penalize angle change between warp; 

       \textbf{bending} springs penalize changes along folds .\\[0.4em]
      \textbf{Energy (schematic).}
      \begin{itemize}
        \item
          \(
            U_{\text{cloth}} = \tfrac{1}{2} \! \sum_{(i,j)\in E} k_s \big(\|X_i - X_j\| - \|x_i - x_j\|\big)^2
          \)
      \end{itemize}
      Differentiation gives \textbf{internal forces} that are linear in displacements around the rest state.\\[0.4em]
      
      }
    \column{0.34\textwidth}
      \vspace*{\fill}
      \begin{flushright}
        \includegraphics[width=1.15\linewidth]{fig2.0.png}\\[-0.3em]
      \end{flushright}
  \end{columns}
  Our 3D \textbf{SMS (for tetrahedra)}  follows the same idea:
      we define simple view of deformation along a few directions and build a \textbf{quadratic energy}, 
      whose gradient gives internal forces and whose Hessian gives a \textbf{global stiffness}. The next pages lift this intuition to 3D tets with anisotropy\cite{bourguignon2000anisotropy,lakhal2013modified}.
\end{frame}

% Slide 2
\begin{frame}
  \frametitle{P2 -- Notation and Problem Setup}
  \begin{itemize}
    \item \textbf{Tetrahedral mesh with $N$ vertices.}
      \[
        \mathbf{x}\in\mathbb{R}^{3N} \text{ (present)}, \quad
        \mathbf{X}\in\mathbb{R}^{3N} \text{ (predicted)}, \quad
        \mathbf{u} = \mathbf{X} - \mathbf{x}.
      \]
    \item \textbf{Dirichlet data from image registration.} On the constrained set $\mathcal D$: $X_i = x_i + u_i$. Free DOFs $\mathcal F$ are unknown.
    \item \textbf{Per-tet notation.} For $\mathcal V_k$, reference vertices $x_0,x_1,x_2,x_3$, barycenter $x_b = \tfrac{1}{4} \sum_{i=0}^3 x_i$.
    \item \textbf{Goal.} Solve for $\mathbf{X}_{\mathcal F}$ so that internal elastic forces balance external loads on the free set.
  \end{itemize}
\end{frame}

% Slide 3
\begin{frame}
  \frametitle{P3 -- Deformation Gradient $F_k$ from Edge Matrices \& SVD Axes}
  We adopt your \textbf{edge-matrix} construction\cite{irving2004invertible}.
  \begin{itemize}
    \item Form $3\times 3$ edge matrices in each tet (local nodes $0,1,2,3$):
      \[
        d_x = \big[x_1 - x_0,\ x_2 - x_0,\ x_3 - x_0\big], \qquad
        D_X = \big[X_1 - X_0,\ X_2 - X_0,\ X_3 - X_0\big].
      \]
    \item Assume the mapping \textbf{present $\to$ predicted} is affine: $X = F x + \text{const}$. Eliminating the translation gives
      \[
        D_X = F\, d_x \quad \Rightarrow \quad \boxed{\,F_k = D_X\, d_x^{-1}\,}.
      \]
      For a nondegenerate tet, $d_x$ is invertible.
    \item \textbf{SVD and axes of anisotropy.} Factor $F_k = U_k \Sigma_k V_k^{\top}$ with $\Sigma_k=\operatorname{diag}(\lambda_{k0},\lambda_{k1},\lambda_{k2})$, $\lambda_{k0}\geq\lambda_{k1}\geq\lambda_{k2}>0$. Because $F_k$ maps the present frame to the predicted one, the columns of $V_k$ span the present (domain) basis, while the columns of $U_k$ live in the predicted frame. Freeze the anisotropy axes in the present frame as
      \[
        \boxed{\,e_{\ell} = (V_k)_{:\ell},\ \ell = 0,1,2\,}
      \]
      and enforce right-handedness if needed (if $(e_0\times e_1)\cdot e_2<0$ flip $e_2$). .
  \end{itemize}
\end{frame}

% Slide 4
\begin{frame}
  \frametitle{P4 -- Barycentric Rays}
  \small
  \begin{columns}[T,onlytextwidth]
    \column{0.56\textwidth}
      \begin{itemize}
        \item \textbf{Barycenter.}
          \[
            x_b = \tfrac{1}{4} \sum_{i=1}^4 x_i .
          \]
        \item \textbf{Axis-aligned rays.} From $x_b$ cast rays along $\pm e_0, \pm e_1, \pm e_2$ (axes from P3) to the tet faces. Each axis intersects twice, producing six points $q_j$ ($j = 1, \dots, 6$).
        \item \textbf{Point-in-triangle test.} If a ray hits face $\Delta_{i_1 i_2 i_3}$,
          \[
            S_{\Delta_{i_1 i_2 i_3}} = S_{\Delta_{q_j i_2 i_3}} + S_{\Delta_{i_1 q_j i_3}} + S_{\Delta_{i_1 i_2 q_j}}.
          \]
      \end{itemize}
    \column{0.36\textwidth}
      \vspace*{\fill}
      \centering
      \hspace*{-0.07\linewidth}\includegraphics[width=1.05\linewidth]{fig2.6.jpg}\\[-0.3em]
  \end{columns}
\end{frame}
% Slide 5
\begin{frame}
  \frametitle{P4 -- Local Coordinates}
  \small
  \begin{itemize}
    \item \textbf{Area coordinates.}
      \[
        \xi = \frac{S_{\Delta_{q_j i_2 i_3}}}{S_{\Delta_{i_1 i_2 i_3}}},\quad
        \eta = \frac{S_{\Delta_{q_j i_1 i_3}}}{S_{\Delta_{i_1 i_2 i_3}}},\quad
        1 - \xi - \eta = \frac{S_{\Delta_{i_1 i_2 q_j}}}{S_{\Delta_{i_1 i_2 i_3}}}.
      \]
    \item \textbf{Coefficient matrix $C^k \in \mathbb{R}^{4 \times 6}$.} Let $N_i$ be the linear shape functions. On the hit face,
      \[
        N_{i_1}(q_j) = 1 - \xi - \eta,\quad N_{i_2}(q_j) = \xi,\quad N_{i_3}(q_j) = \eta,\quad N_{i_4}(q_j) = 0.
      \]
    \item \textbf{Affine updates.} Define $C^k_{ij} = N_i(q_j)$ for $i = 1..4$ and $j = 1..6$. With current vertices $X_i^t$,
      \[
        X_j^t = \sum_{i=1}^4 C^k_{ij} \, X_i^t .
      \]
  \end{itemize}
\end{frame}

% Slide 6
\begin{frame}
  \frametitle{P5 -- Energy Definition}
  \normalsize
  \begin{columns}[T,onlytextwidth]
    \column{0.64\textwidth}
      \textbf{Full per-tet energy (finite-strain with fixed axes).}
      {\small\[
        \begin{aligned}[t]
        U_k(F_k)=V_k\!\big[\;&
          \tfrac{\alpha_k}{2}\sum_{\ell=0}^{2}\big(\|F_k e_{\ell}\|^2-1\big)^2\\
          &+\tfrac{\beta_k}{2}\sum_{\ell<m}\big((F_k e_{\ell})\!\cdot\!(F_k e_m)\big)^2\\
          &+\tfrac{\kappa_k}{2}(J_k-1)^2\big]
        \end{aligned}
      \]}
      where $F_k$ is the deformation gradient from the current nodal positions and $J_k = \det F_k$.

      
    \column{0.28\textwidth}
      \vspace*{\fill}
      \centering
      \includegraphics[width=1\linewidth]{fig2.12.jpg}
  \end{columns}
      \begin{itemize}
        \item \emph{Axial.} $\|F_k e_{\ell}\|^2 = e_{\ell}^{\top} C_k e_{\ell}$ with $C_k = F_k^{\top}F_k$. At rest $C_k = I$, so the term vanishes; with $F_k=I+H$ it linearises to $2\,\varepsilon_{\ell\ell}$, i.e., stretch/compression along $e_{\ell}$.
        \item \emph{Shear.} $(F_k e_{\ell})\!\cdot\!(F_k e_m) = e_{\ell}^{\top} C_k e_m$ is the off-diagonal metric component in the $\{e_{\ell}\}$ frame. Small strain gives $2\,\varepsilon_{\ell m}$, capturing loss of orthogonality between axes.
        \item \emph{Volume.} $J_k - 1$ measures relative volume change and reduces to $\operatorname{tr}\varepsilon$ in the linear regime.
      \end{itemize}

\end{frame}

% Slide 7
\begin{frame}
  \frametitle{P5 -- SMS Stiffness \& FEM Correspondence}
  \normalsize

  \textbf{Small-strain quadratic approximation (SMS).}
      {\small\[
        w_{\text{quad}}(\varepsilon)=2\alpha_k\!\sum_{\ell}\varepsilon_{\ell\ell}^2
        +2\beta_k\!\sum_{\ell<m}\varepsilon_{\ell m}^2
        +\tfrac{\kappa_k}{2}(\operatorname{tr}\varepsilon)^2,\qquad
        U_k\approx V_k\,w_{\text{quad}}(\varepsilon_k).
      \]}
  \textbf{Comparison with continuous FEM.}
  \[
    W_{\text{FEM}}(\varepsilon)=\mu\,\varepsilon:\varepsilon+\tfrac{\lambda}{2}(\operatorname{tr}\varepsilon)^2
  \]
  Choosing $\alpha=\mu/2$, $\beta=\mu$, $\kappa=\lambda$ recovers the standard bilinear form $a(u,v)=\int\sigma(u):\varepsilon(v)\,dx$ for $P_1$ FEM.
\end{frame}

% Slide 8
\begin{frame}
  \frametitle{P6 -- Forward Solver (Free DOFs)}
  \normalsize
  \begin{enumerate}
    \item \textbf{Assemble stiffness $K_{FF}$.} Loop over tets, add the three $r$-outer-product kernels with coefficients $4\alpha_k V_k$, $4\beta_k V_k$, $\kappa_k V_k$, then restrict to free DOFs via the dof map.
    \item \textbf{Assemble RHS $b_F$.} Lump tet mass ($\rho_k V_k/4$ per vertex), multiply by gravity.
    \item \textbf{Solve the SPD system}
      \[
        K_{FF}\,u_F=b_F
      \]
      with a sparse solver; recover $u$ by inserting zeros on Dirichlet DOFs and set $X = x + u$.
    \item \textbf{Notes.} Keep consistent units (mesh in m, $E$ in Pa, $\rho$ in kg/m$^3$). Axes $\{e_{\ell}\}$ remain fixed during optimization.
  \end{enumerate}
\end{frame}

% References
\begin{frame}[allowframebreaks]
  \frametitle{References}
  \bibliographystyle{plainnat}
  \bibliography{refs}
\end{frame}

\end{document}
